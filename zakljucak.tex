Potreba za dodatnim memorijskim prostorom postoji od kad postoje računala, a sad je još izraženija zbog silne količine podataka koje nam stižu svakodnevno. Isto vrijedi i za područje bioinformatike gdje se svakim danom sakuplja sve više različitih, ali i sličnih sekvenci. Upravo u toj sličnosti su tvorci originalnog rada na temu HRCM-a pronašli rješenje za problem memorije. Zašto ponovo unositi i spremati iste dijelove sekvence koji postoje u nekoj drugoj već spremljenoj sekvenci? Zašto ponovo spremiti cijelu sekvencu kad slična već postoji u našoj memoriji? Samo ju malo promijenimo. Ovo je jako lukav potez. Proći po sekvencama i samo reći po čemu se jedna razlikuje od druge u relativno par crtica koje zauzimaju djelić prostora u usporedbi s cijelim sekvencama. Naravno postupak nije savršen, a može se postići čak i suprotan učinak ako sekvence koje su korištene u algoritmu nisu dovoljno velike ili slične. Sam postupak je u trenucima dosta logički zahtjevan i postoji puno mjesta na kojima se može pogriješiti, ali ako se dobro izvede i nad pravim podacima učinak je itekako vidljiv. 